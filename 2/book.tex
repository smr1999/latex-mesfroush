%%% Sayyed Mohammad Rouzegar %%%

\documentclass[11pt]{book} 
% Just accept range 10pt to 12 pt

\usepackage[colorlinks=true,linkcolor=blue,linkbordercolor={1 1 1}]{hyperref}

\usepackage{xepersian} % be carefull this is the last package we should import!

\settextfont[    
Scale=1.09,
Extension=.ttf, 
Path=fonts/,
ItalicFont=IRNazaninItalic,
BoldFont=IRNazaninBold
]{IRNazanin}

%\settextfont[Scale=1.09,Extension=.ttf,Path=fonts/]{IranNastaliq}

\defpersianfont\nast[Scale=1,Extension=.ttf,Path=fonts/]{IranNastaliq}

\defpersianfont\iransans[Scale=1,Extension=.ttf,Path=fonts/,BoldFont=IranSansBold]{IranSans}

% use macros for half-space -> editor.write("\u200C") with shorcuts shift+space

\title{متن آزمایشی}
\author{سیدمحمد روزگار}
% date{} %use for blank date
\date{20 شهریور 1390}

\begin{document}
	\maketitle

\chapter{فصل اول} \label{chapter1}
لورم ایپسوم متن ساختگی با تولید سادگی نامفهوم از صنعت چاپ و با استفاده از طراحان گرافیک است. چاپگرها و متون بلکه روزنامه و مجله در ستون و سطرآنچنان که لازم است و برای شرایط فعلی تکنولوژی مورد نیاز و کاربردهای متنوع با هدف بهبود ابزارهای کاربردی می باشد. کتابهای زیادی در شصت و سه درصد گذشته، حال و آینده شناخت فراوان جامعه و متخصصان را می طلبد تا با نرم افزارها شناخت بیشتری را برای طراحان رایانه ای علی الخصوص طراحان خلاقی و فرهنگ پیشرو در زبان فارسی ایجاد کرد. در فصل \hyperref[chapter2]{\ref{chapter2}} خواهید دید که ...

\chapter{فصل دوم} \label{chapter2}
در این صورت می توان امید داشت که تمام و دشواری موجود در ارائه راهکارها و شرایط سخت تایپ به پایان رسد وزمان مورد نیاز شامل حروفچینی دستاوردهای اصلی و جوابگوی سوالات پیوسته اهل دنیای موجود طراحی اساسا مورد استفاده قرار گیرد.
\end{document}