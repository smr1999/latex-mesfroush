\documentclass[12pt]{article}

\usepackage[colorlinks=true,linkcolor=blue,linkbordercolor={1 1 1}]{hyperref}

% Use this package for symbols and math equations
\usepackage{amsmath,amssymb,amsfonts}

\usepackage{breqn}

\usepackage{xepersian}

\settextfont[Path=fonts/,Extension=.ttf,BoldFont=YasBd,ItalicFont=YasIt,BoldItalicFont=YasBdIt]{Yas}

% If we want our digits in perian
\setdigitfont[Path=fonts/,Extension=.ttf,BoldFont=YasBd,ItalicFont=YasIt,BoldItalicFont=YasBdIt]{Yas}

\setcounter{secnumdepth}{4}
\setcounter{tocdepth}{4}

\newcommand{\pde}{معادله جزئئ با مشتقات جزئی \ }
% needs two args
\newcommand{\mian}[2]{\frac{#1_1 + #1_2 + \cdots + #1_{#2}}{#2}}

\title{متن آزمایشی} 
\author{سید محمد روزگار}

\begin{document}
\maketitle

فرمول نویسی

% Do not use empty newline in \[ equation \]
% With \quad and \qquad and \ (\space) and \, and \; and \: we can set space in equations
% Use \! for space in reverse like 
\[
a = b + c,\quad d=e-f\\
e = mc^2 ,\qquad e = mgh
e = mc^2 ,\; e = mgh
e = mc^2 ,\qquad e = a\!+ b 
\]

% For little symbols 
\[
\alpha,\beta,\gamma,\omega,\eta,\rho,\phi,\theta,\nu,\kappa,\sigma,\int,\phi,\psi
\]

% For large symbols
\[
\Sigma,\Theta,\Phi,\Psi
\]

% Use \boldmath for bold ...
\[
\Sigma,\boldsymbol{\Sigma}
\]

% Use var in symbols
\[
\phi,\varphi,\epsilon,\varepsilon
\]

% Power(^) and sub(_) 
\[
a = A^{XYZ},\quad b= B_{XYZ},\qquad c = a^{22}+b^{2},\alpha^{\gamma} = \beta^{100} = c^9
\]

\[
\sin^2x + \cos^2x=1,\quad \sec^2x+\csc^2x
\]

\[
x_1 + x_{11} + x_{1+j}
\]

% use \mathbf{text} for bold nosymbols

\[
\mathbf{x} = (x_1+x_2+\cdots+x_{100})
\]

% Use \sqrt[root]{arg} for radical
\[
a = \sqrt[5]{x^2+y^2},\quad b = \sqrt{a+b-c\quad}
\]

\[
\mathbf{a} = \sqrt[4]{3+\sqrt{86+\sqrt[3]{x+1}}}
\]

% Use \frac{num}{den} for fraction
\[
a = \frac{x+1}{x^2-3}
\]

\[
\frac{\sin^2(\alpha - 1)+x^2}{\sqrt{x^\beta-2\frac{m^2+n_1}{999}}}
\]

% We can use \tfrac{num}{den} instead of \frac{num}{den} but smaller
\[
\tfrac{\sin^2(\alpha - 1)+x^2}{\sqrt{x^\beta-2\tfrac{m^2+n_1}{999}}}
\]

\[
\frac12 \frac{x-1}{x+2}
\]

% Use log_b for log in base b
\[
\sqrt{\tfrac{1}{k}\log_bx}
\]

% use dots with \cdots , \dots , \ldots -> \dots justify dots
\[
\mathbf{x} = (x_1+x_2+\cdots+x_{100}),\quad \mathbf{y} = (x_1+x_2+\dots+x_{100}),\quad \mathbf{y} = (x_1+x_2+\ldots+x_{100})
\]

% use vertical dots with \vdots and diagonal dots with \ddots
\[
x \vdots y,\qquad x \ddots y
\]

% Use dot with \cdot
\[
x = A.B,\quad y = A \cdot B
\]

% Use \int for antegral
\[
A = \int (x^2+2x-1)\,\mathrm{d}x
\]

% Use \int with lower and upper bound
\[
B = \int_{-\infty}^{\infty} \sin^2x\,\mathrm{d}x,
\]

% Multiple antegral
\[
C = \iiint a^2+b^2 \, \mathrm{d}x,\quad C = \idotsint a^2+b^2 \, \mathrm{d}x
\]

\[
D = \int_{0}^{10} \int_{1}^{100} f(x,y)\,\mathrm{d}x\,\mathrm{d}y
\]

\[
\int_{0}^{\infty}\frac{\sin{x}}{\sqrt{2x-\ln x+1}}\,\mathrm{d}x
\]

% We can use limits with \lim
\[
\lim f(x),\quad \lim_{x \to 0 } \frac{x-1}{x^{2}-1}
\]

\[
\lim_{x \to \frac{\pi}{2}} \int_0^t \frac{x-1}{\cos x+ \tan x}\, \mathrm{d}x
\]

% use bar
\[
\mathbf{\bar{x}} = \frac{x_1+x_2+\dots+x_n}{n}
\]

%use hat
\[
\hat{x} = a+b
\]

% use wide hat
\[
\widehat{xyz} = a+b
\]

% use tilde
\[
\tilde{x} = a-b
\]

% use wide tilde
\[
\widetilde{wxyz} = a^2-b
\]

% use \overline{text} instead of wide bar
\[
\overline{wxyza} = \frac{2}{3},\quad \underline{wxyza}
\]

% use \dot
\[
\dot{u} + 2u = f
\]

% use \ddot , \dddot and \ddddot
\[
\ddot{u} - \Delta u = f, \dddot{u}, \ddddot{u}
\]

% use $\vec
\[
\vec{x} = a^2
\]

% use \overleftarrow{text} , \overrightarrow{text} and \overleftrightarrow{argument}
\[
\overleftarrow{qwert} = a+b ,\; \overrightarrow{trewq} = a-b,\; \overleftrightarrow{qwert} = a \times b
\]

% use \underleftarrow{argument} and \underrightarrow{argument}
\[
 \underleftarrow{qwert} = a+b ,\; \underrightarrow{trewq} = a-b
\]

% use \overbrace{text} 
\[
\bar{x} = \frac{\overbrace{x_1 + x_2 + \cdots + x_n}^{n \ \mathrm{times}}}{n}
\]

% use \text to use persian
\[
\bar{x} = \frac{\overbrace{x_1 + x_2 + \cdots + x_n}^{\text{بار} \ n}}{n}
\]

% use \underbrace{text}
\[
\underbrace{y_1 \cdot y_2 \cdots y_m}_{\text{مرتبه} \ m}
\]

% use \xrightarrow and \xleftarrow
\[
\lim_{x \to 0} f(x) = 0 \xrightarrow{\text{بر اساس قضیه}} B,\quad \lim_{x \to 0} f(x) = 0 \xleftarrow{\text{بر اساس قضیه}} B
\]

\[
\lim_{x \to 0} f(x) = 0 \xrightarrow[we\ know\ that\ \cdots]{\text{بر اساس قضیه}} B
\]

% use sum
\[
\sum,\ \sigma,\ \Sigma
\]
\[
\sum_{n = 1}^{n = \infty}  \frac{n_1 + n_2 + \cdots + n_k}{k}
\]

% use product
\[
\Pi,\pi,\prod
\]
\[
\prod_{i=1}^{n} \frac{x-x_i}{x_i+x_j}
\]

% use sideset
\[
\sideset{_{x=1}^{100}}{_{y=1}^{1000}}{\prod} x_{i} \cdot y_{i}
\]

% use \overset and \joinref and \ast
\[
\lim_{x \to 0} f(x) \overset{Hop}{= \joinrel =} 0,\quad \overset{\ast\ast}{X}
\]

% use \overset
\[
\underset{*}{Y}
\]

\[
\overset{a}{\underset{b}{X}}
\]

% use \neq
\[
L_i(x) = \prod_{j=0,i \neq j}^{n} \frac{x-x_j}{x_i - x_j}
\]

% use \substack{argument}
\[
L_i(x) = \prod_{\substack{j=0\\i \neq j}}^{n} \frac{x-x_j}{x_i - x_j}
\]

\[
\sum_{j=k}^{\substack{100\\ k \neq 10}}anything
\]

\rule{\textwidth}{1pt}

روش انتگرال گیری جزء به جزء برای 
% in line formula
$
\int u\ \mathrm{d}v
$
به شکل روبرو است : 
$
\int \sin x + \cos x \ \mathrm{d}x
$

% use equations with numbers -> helps us to refers to them

\section{مقدمات}

\begin{equation} \label{eq1}
\int \sin x + \cos x \ \mathrm{d}x
\end{equation}

\begin{equation} \label{eq2}
L_i(x) = \prod_{j=0,i \neq j}^{n} \frac{x-x_j}{x_i - x_j}
\end{equation}


\begin{equation} \label{eq3}
\lim_{x \to 0} f(x) \overset{Hop}{= \joinrel =} 0,\quad \overset{\ast\ast}{X}
\end{equation}

با توجه به فرمول 
% use \eqref{label} instead of \ref for refrence to equations
\eqref{eq1}
می دانیم :
\section{انواع فرمول های یکسان}
\[
L_i(x) = \prod_{j=0,i \neq j}^{n} \frac{x-x_j}{x_i - x_j}
\]
$$
L_i(x) = \prod_{j=0,i \neq j}^{n} \frac{x-x_j}{x_i - x_j}
$$
\begin{equation*}
	L_i(x) = \prod_{j=0,i \neq j}^{n} \frac{x-x_j}{x_i - x_j}
\end{equation*}

\section{فرمول نویسی چند خطی}
% use % to justify left and right content
\begin{align*}
a & = b \\
& =a^2+b\\
& =\sin x
\end{align*}

\begin{align*}
L_i(x) &= \prod_{\substack{j=0\\j \neq i}}^{n} \frac{x-x_j}{x_i-x_j} \\
& = \frac{(x-x_0)(x-x_1)\cdots (x-x_n)}{(x_i - x_1)\cdots (c_i - x_n)}
\end{align*}

% To have number equation use align instead of align*
% use \notag to have not number equation
\begin{align}
	L_i(x) &= \prod_{\substack{j=0\\j \neq i}}^{n} \frac{x-x_j}{x_i-x_j} \notag\\
	& = \frac{(x-x_0)(x-x_1)\cdots (x-x_n)}{(x_i - x_1)\cdots (c_i - x_n)}\label{eq_imp}
\end{align}

همانطور که در رابطه
\eqref{eq_imp} 
دیدیم ...

\section{اگر چند فرمول داشتیم که همگی می خواستیم یک شماره در وسطش بخورد}
% to use this equation and split
\begin{equation}\label{eq_total}
\begin{split}
F(x) & = \int_0^1 f(x)\mathrm{d} x\\
& F(b) - F(a)\\
& \le F(x) \\
& = 12
\end{split}
\end{equation}

% use dgroup and dmath
\begin{dgroup}

\begin{dmath}
\sin^2 x + \cos^2 x = 1
\end{dmath}

\begin{dmath}
\tan x \cot x =1
\end{dmath}

\begin{dmath}
\sin x \cos x
\end{dmath}

\end{dgroup}

% use Normal an Real symbols with \mathbb{text}
\[
\mathbb{R},\quad \mathbb{N}
\]

% use analysis symbols with $\mathcal{text}$
\[
\mathcal{A},\quad \mathcal{G}
\]

\section{تولید ماتریس}
% create matrix with matrix enviroment
\begin{align*}
\begin{matrix}
a_{1,1} & a_{1,2} & a_{1,3}\\
a_{2,1} & a_{2,2} & a_{2,3}\\
\end{matrix}
\end{align*}

% create another matrix with bmatrix enviroment
\begin{align*}
	\begin{bmatrix}
		a_{1,1} & a_{1,2} & a_{1,3}\\
		a_{2,1} & a_{2,2} & a_{2,3}\\
	\end{bmatrix}
\end{align*}

% create another matrix with Bmatrix enviroment
\begin{align*}
	\begin{Bmatrix}
		a_{1,1} & a_{1,2} & a_{1,3}\\
		a_{2,1} & a_{2,2} & a_{2,3}\\
	\end{Bmatrix}
\end{align*}

% create another matrix with vmatrix enviroment
\begin{align*}
	\begin{vmatrix}
		a_{1,1} & a_{1,2} & a_{1,3}\\
		a_{2,1} & a_{2,2} & a_{2,3}\\
	\end{vmatrix}
\end{align*}


% create another matrix with Vmatrix enviroment
\begin{align*}
	\begin{Vmatrix}
		a_{1,1} & a_{1,2} & a_{1,3}\\
		a_{2,1} & a_{2,2} & a_{2,3}\\
	\end{Vmatrix}
\end{align*}

% use small matrix
در ماتریس 
$
\begin{smallmatrix}
\begin{bmatrix}
1&2&3\\
4&5&6\\
7&8&9
\end{bmatrix}
\end{smallmatrix}
$
داریم ...\\

\rule{\textwidth}{1pt}\\

\begin{align*}
A = 
\begin{bmatrix}
	\frac{1}{2} & 1 & 2 \\
	3 & 2 & \frac{2}{3} \\
	5 & 0 & 0	
\end{bmatrix}
\cdot
\begin{bmatrix}
	4 & \frac{1}{2} & 0 \\
	-1 & 3 & 2\\
	2 & \frac{1}{12} & 4
\end{bmatrix}
\end{align*}

\[
\begin{bmatrix}
a_{1,1} & a_{1,2} & a_{1,3} & \cdots & a_{1,n} \\
a_{2,1} & a_{2,2} & a_{2,3} & \cdots & a_{2,n} \\
\vdots & \vdots & \ddots & \vdots & \vdots \\
a_{m-1,1} & a_{m-1,2} & a_{m-1,3} & \cdots & a_{m-1,n} \\
a_{m,1} & a_{m,2} & a_{m,3} & \cdots & a_{m,n} 
\end{bmatrix}
\]

\section{توابع چند ضابطه ای}

% use cases enviroment for this functions 

\begin{equation}
\mathbf{|x|} =
\begin{cases}
	x & x\ge 0,\\
	-x & x < 0	
\end{cases}
\end{equation}

% use \lbrace and \rbrace or \{ and \}
\[
A = \lbrace v \rbrace, \{ v \}
\]

% use \colon
\[
A = \lbrace v : \text{پیوسته است} \ v \rbrace
\],\quad 
\[
A = \lbrace v \colon \text{پیوسته است} \ v \rbrace
\]

\[
|a| , ||a||,\| a \| , \lvert a \rvert, \lVert a \rVert
\]

\section{استفاده از کامندها}

\pde زیر را در نظر بگیرید
. در این \pde می خواهیم ...

\[
\bar{A} = \mian{x}{10} , \quad \bar{B} = \mian{y}{100}
\]

\end{document}