\documentclass[12pt]{article}

\usepackage[colorlinks=true,linkcolor=blue,linkbordercolor={1 1 1}]{hyperref}

% Use this package for symbols and math equations
\usepackage{amsmath,amssymb}

\usepackage{xepersian}

\settextfont[Path=fonts/,BoldFont=IRNazaninBold,ItalicFont=IRNazaninItalic,Extension=.ttf]{IRNazanin}

% If we want our digits in perian
\setdigitfont[Path=fonts/,BoldFont=IRNazaninBold,ItalicFont=IRNazaninItalic,Extension=.ttf]{IRNazanin}


\setcounter{secnumdepth}{4}
\setcounter{tocdepth}{4}

\title{متن آزمایشی}
\author{سید محمد روزگار}

\begin{document}
\maketitle

فرمول نویسی

% Do not use empty newline in \[ equation \]
% With \quad and \qquad and \ (\space) and \, and \; and \: we can set space in equations
% Use \! for space in reverse like 
\[
a = b + c,\quad d=e-f\\
e = mc^2 ,\qquad e = mgh
e = mc^2 ,\; e = mgh
e = mc^2 ,\qquad e = a\!+ b 
\]

% For little symbols 
\[
\alpha,\beta,\gamma,\omega,\eta,\rho,\phi,\theta,\nu,\kappa,\sigma,\int,\phi,\psi
\]

% For large symbols
\[
\Sigma,\Theta,\Phi,\Psi
\]

% Use \boldmath for bold ...
\[
\Sigma,\boldsymbol{\Sigma}
\]

% Use var in symbols
\[
\phi,\varphi,\epsilon,\varepsilon
\]

% Power(^) and sub(_) 
\[
a = A^{XYZ},\quad b= B_{XYZ},\qquad c = a^{22}+b^{2},\alpha^{\gamma} = \beta^{100} = c^9
\]

\[
\sin^2x + \cos^2x=1,\quad \sec^2x+\csc^2x
\]

\[
x_1 + x_{11} + x_{1+j}
\]

% use \mathbf{text} for bold nosymbols

\[
\mathbf{x} = (x_1+x_2+\cdots+x_{100})
\]

% Use \sqrt[root]{arg} for radical
\[
a = \sqrt[5]{x^2+y^2},\quad b = \sqrt{a+b-c\quad}
\]

\[
\mathbf{a} = \sqrt[4]{3+\sqrt{86+\sqrt[3]{x+1}}}
\]

% Use \frac{num}{den} for fraction
\[
a = \frac{x+1}{x^2-3}
\]

\[
\frac{\sin^2(\alpha - 1)+x^2}{\sqrt{x^\beta-2\frac{m^2+n_1}{999}}}
\]

% We can use \tfrac{num}{den} instead of \frac{num}{den} but smaller
\[
\tfrac{\sin^2(\alpha - 1)+x^2}{\sqrt{x^\beta-2\tfrac{m^2+n_1}{999}}}
\]

\[
\frac12 \frac{x-1}{x+2}
\]

% Use log_b for log in base b
\[
\sqrt{\tfrac{1}{k}\log_bx}
\]

% use dots with \cdots , \dots , \ldots -> \dots justify dots
\[
\mathbf{x} = (x_1+x_2+\cdots+x_{100}),\quad \mathbf{y} = (x_1+x_2+\dots+x_{100}),\quad \mathbf{y} = (x_1+x_2+\ldots+x_{100})
\]

% use vertical dots with \vdots and diagonal dots with \ddots
\[
x \vdots y,\qquad x \ddots y
\]

% Use dot with \cdot
\[
x = A.B,\quad y = A \cdot B
\]

% Use \int for antegral
\[
A = \int (x^2+2x-1)\,\mathrm{d}x
\]

% Use \int with lower and upper bound
\[
B = \int_{-\infty}^{\infty} \sin^2x\,\mathrm{d}x,
\]

% Multiple antegral
\[
C = \iiint a^2+b^2 \, \mathrm{d}x,\quad C = \idotsint a^2+b^2 \, \mathrm{d}x
\]

\[
D = \int_{0}^{10} \int_{1}^{100} f(x,y)\,\mathrm{d}x\,\mathrm{d}y
\]

\[
\int_{0}^{\infty}\frac{\sin{x}}{\sqrt{2x-\ln x+1}}\,\mathrm{d}x
\]

\end{document}